\documentclass[a4paper]{article}
\usepackage[english]{babel}
\usepackage[a4paper,top=2cm,bottom=2cm,left=2cm,right=2cm,marginparwidth=1.75cm]{geometry}
\usepackage{amsmath}
\usepackage{amsfonts}
% \usepackage{amsthm}
\usepackage{amssymb}
\usepackage{graphicx}
\usepackage[colorinlistoftodos]{todonotes}
\usepackage[colorlinks=true, allcolors=blue]{hyperref}
\usepackage{import}
\usepackage{pdfpages}
\usepackage{transparent}
\usepackage{xcolor}
\usepackage{algorithmicx}
\usepackage{algpseudocode}

\usepackage{thmtools}
\usepackage{enumitem}
\usepackage[framemethod=TikZ]{mdframed}

\usepackage{xpatch}

\usepackage{boites}
\makeatletter
\xpatchcmd{\endmdframed}
{\aftergroup\endmdf@trivlist\color@endgroup}
{\endmdf@trivlist\color@endgroup\@doendpe}
{}{}
\makeatother

%\usepackage[poster]{tcolorbox}
%\allowdisplaybreaks
%\sloppy

\usepackage[many]{tcolorbox}

\xpatchcmd{\proof}{\itshape}{\bfseries\itshape}{}{}

% to set box separation
\setlength{\fboxsep}{0.8em}
\def\breakboxskip{7pt}
\def\breakboxparindent{0em}

\newenvironment{proof}{\begin{breakbox}\textit{Proof.}}{\hfill$\square$\end{breakbox}}
\newenvironment{ans}{\begin{breakbox}\textit{Answer.}}{\end{breakbox}}
\newenvironment{soln}{\begin{breakbox}\textit{Solution.}}{\end{breakbox}}

% \tcolorboxenvironment{proof}{
%     blanker,
%     before skip=\topsep,
%     after skip=\topsep,
%     borderline={0.4pt}{0.4pt}{black},
%     breakable,
%     left=12pt,
%     right=12pt,
%     top=12pt,
%     bottom=12pt,
% }
%
% \tcolorboxenvironment{ans}{
%     blanker,
%     before skip=\topsep,
%     after skip=\topsep,
%     borderline={0.4pt}{0.4pt}{black},
%     breakable,
%     left=12pt,
%     right=12pt,
% }

\mdfdefinestyle{enclosed}{
    linecolor=black
    ,backgroundcolor=none
    ,apptotikzsetting={\tikzset{mdfbackground/.append style={fill=gray!100,fill opacity=.3}}}
    ,frametitlefont=\sffamily\bfseries\color{black}
    ,splittopskip=.5cm
    ,frametitlebelowskip=.0cm
    ,topline=true
    ,bottomline=true
    ,rightline=true
    ,leftline=true
    ,leftmargin=0.01cm
    ,linewidth=0.02cm
    ,skipabove=0.01cm
    ,innerbottommargin=0.1cm
    ,skipbelow=0.1cm
}

\mdfsetup{%
    middlelinecolor=black,
    middlelinewidth=1pt,
roundcorner=4pt}

\setlength{\parindent}{0pt}

\mdtheorem[style=enclosed]{theorem}{Theorem}
\mdtheorem[style=enclosed]{lemma}{Lemma}[theorem]
\mdtheorem[style=enclosed]{claim}{Claim}[theorem]
\mdtheorem[style=enclosed]{ques}{Question}
\mdtheorem[style=enclosed]{defn}{Definition}
\mdtheorem[style=enclosed]{notn}{Notation}
\mdtheorem[style=enclosed]{obs}{Observation}
\mdtheorem[style=enclosed]{eg}{Example}
\mdtheorem[style=enclosed]{cor}{Corollary}
\mdtheorem[style=enclosed]{note}{Note}

% \let\thetheorem=\relax
% \let\thelemma=\relax
% \let\theclaim=\relax
% \let\theques=\relax
% \let\thedefn=\relax
% \let\thenotn=\relax
% \let\theobs=\relax
% \let\thecor=\relax
% \let\thenote=\relax

% \renewcommand\qedsymbol{$\blacksquare$}
\newcommand{\nl}{\vspace{0.2cm}\\}
\newcommand{\ol}{\overline}
\newcommand{\eps}{\varepsilon}
\newcommand{\mc}{\mathcal}
\newcommand{\mi}{\mathit}
\newcommand{\mf}{\mathbf}
\newcommand{\mb}{\mathbb}
\newcommand{\R}{\mathbb{R}}
\newcommand{\Z}{\mathbb{Z}}
\newcommand{\OPT}{\mathbf{OPT}}
\newcommand{\ALG}{\mathbf{ALG}}
\renewcommand{\L}{\mc{L}}
\newcommand{\changesto}{\vdash}
\newcommand\Vtextvisiblespace[1][.3em]{%
    \mbox{\kern.06em\vrule height.3ex}%
    \vbox{\hrule width#1}%
    \hbox{\vrule height.3ex}
}
\newcommand{\blank}{{\Vtextvisiblespace[0.7em]}}
\newcommand{\leftend}{\triangleright}
\newcommand{\comp}{\overline}

\newcommand{\incfig}[1]{%
    \def\svgwidth{\columnwidth}
    \import{./figures/}{#1.pdf_tex}
}
\pdfsuppresswarningpagegroup=1

\title{\textbf{Approximation Algorithms Lecture 13}}
\date{}

%\section{Recap}
%
%\section{Definitions}
%
%\begin{defn}
%\end{defn}
%
%\section{Content}
%
%\begin{theorem}
%\end{theorem}
%\begin{proof}
%\end{proof}
%
%\begin{ques}
%\end{ques}
%
%\begin{eg}
%\end{eg}
%
%\begin{claim}
%\end{claim}

\begin{document}
\maketitle
\tableofcontents

\section{Recap}

APTAS for bin packing.

\section{Content}

\subsection{Linear program for vertex cover}
Our template would be the following: make an integer program, do a suitable relaxation to get a linear program, then solve it to get an optimal solution of the LP, then round this fractional solution
to get an integer solution, which could be a solution to the original problem too.\nl
So for vertex cover, we have a graph $G = (V, E)$, and we want to pick a set of vertices, which covers all edges.\nl
Let $x_v = 1$ if $v \in V$ is in the vertex cover, and $0$ otherwise.\nl
What should the constraints on these variables be to ensure that the vertices $v$ with $x_v = 1$ form a vertex cover?\nl
$\forall e = (u, v) \in E: x_u + x_v \ge 1$.\nl
The objective function is $\sum_{v \in V} x_v$.\nl
So the integer program becomes something like the following:

\begin{align*}
    &\min \sum_{v \in V} x_v\\
    &\forall e = (u, v) \in E: x_u + x_v \ge 1\\
    &\forall v \in V: x_v \in \{0, 1\}
\end{align*}

Note that the integer program always corresponds to a vertex cover.\nl

Relaxing this to a linear program gives us:

\begin{align*}
    &\min \sum_{v \in V} x_v\\
    &\forall e = (u, v) \in E: x_u + x_v \ge 1\\
    &\forall v \in V: 0 \le x_v \le 1
\end{align*}

\begin{claim}
    The constraint $x_v \le 1$ is redundant.
\end{claim}
\begin{proof}
    Replace all $x_v > 1$ by $x_v = 1$, and note that it doesn't violate any constraint, but only decreases the value of the objective function.
\end{proof}

Hence, the linear program is equivalent to

\begin{align*}
    &\min \sum_{v \in V} x_v\\
    &\forall e = (u, v) \in E: x_u + x_v \ge 1\\
    &\forall v \in V: x_v \ge 0
\end{align*}

Let's try to write the dual program of this.\nl

\begin{note}
    Note that the canonical forms are the following:
    \begin{align*}
        &\max c^\top x\\
        &Ax \le b\\
        &x \ge 0
    \end{align*}
    and
    \begin{align*}
        &\min c^\top x\\
        &Ax \ge b\\
        &x \ge 0
    \end{align*}
\end{note}

To get the multipliers, we do the following: the objective function has coefficient of $u$ being $1$.\nl
So the dual is:

\begin{align*}
    &\min \sum_{e \in E} y_e\\
    &\forall u \in V: \sum_{e \in \delta(u)} y_e \le 1\\
    &\forall e \in E: y_e \ge 0
\end{align*}

where $\delta(u)$ is the set of edges incident to a vertex $u$.\nl

Suppose primal is a minimization problem, then dual is a maximization problem.\nl
So the value of the dual will always be on the left of the value of the primal.\nl

Now let's come back to the primal.

\begin{align*}
    &\min \sum_{v \in V} x_v\\
    &\forall e = (u, v) \in E: x_u + x_v \ge 1\\
    &\forall v \in V: x_v \ge 0
\end{align*}

Let $x^*$ be the optimal solution to this linear program.\nl
Then we have $\sum_{v \in V} {x^*_v} \le \OPT_{VC}$, since we have only relaxed the constraints.
Note that $x^*$ is not integral, i.e., it could be fractional.\nl
(For bipartite, $x^*$ is integral, and in general it is half-integral, and the dual corresponds to the maximum independent set).\nl
We round $x^*$ to an integral solution as follows: if $x*_v \ge 1/2$, round it to $1$, else to $0$.\nl

Is $\ol{x}$ a feasible solution to the integer program? Yes, since if $\ol{x}_v = 0$ and $\ol{x}_u = 0$, then $x^*_v < \frac{1}{2}$ and $x^*_u < \frac{1}{2}$, which would violate the
inequality.\nl

Now let's look at the quality of the solution. We have the size of the vertex cover as:

\begin{align*}
    \sum_{v \in V} \ol{x}_v &\le \sum_{v \in V} 2 x^*_v\\
                            &\le 2\cdot \OPT_{VC}
\end{align*}

This gives us a 2-approximation.\nl

Now suppose we have weights on the vertices, i.e., a function $w : V \to \R^+$, and we want a minimum weight vertex cover.\nl
We need to just change the objective function by replacing $x_v$ by $w(v) x_v$, and the rest of the proof holds.\nl

Let's look at what the dual program becomes.
We just need to change the RHS from $1$ to $w(u)$.

\subsection{Linear program for set cover}

Let $U = \{e_1, \ldots, e_n\}$, and $S_1, \ldots, S_m \subseteq U$.\nl
$x_i$ is the variable corresponding to $S_i$ such that $x_i = 1$ if $S_i$ is in the set cover, and 0 otherwise.\nl
Now we need: $\forall e_j \in U: \sum_{i: e_j \in S_i} x_i \ge 1$, and $x_i \in \{0, 1\} \forall i$.\nl
Our objective function is $\sum_{i} x_i$.\nl

The corresponding integer program is:

\begin{align*}
    &\min \sum_{i} x_i\\
    &\forall e_j \in U: \sum_{i : e_j \in S_i} x_i \ge 1\\
    &x_i \in \{0, 1\} \forall i
\end{align*}

By the same argument as before, we can relax it to:

\begin{align*}
    &\min \sum_{i} x_i\\
    &\forall e_j \in U: \sum_{i : e_j \in S_i} x_i \ge 1\\
    &x_i \ge 0 \forall i
\end{align*}

Consider the dual of this. Coefficient of $x_i$ in the linear combination should not exceed the coefficient of $x_i$ in the objective function.

\begin{align*}
    &\max \sum_j y_j\\
    &\forall i \quad \sum_{j \mid e_j \in S_i} y_j \le 1\\
    &\forall j \quad y_j \ge 0
\end{align*}

Note that this aligns precisely with the argument we were trying to make for the set cover problem.\nl

We are doing something like this: relax integer program to linear program, then make a feasible solution to the dual program, and use that as an approximation.\nl

\subsection{Set cover through LP rounding}

\begin{enumerate}
    \item Solve the primal LP to get $x^*$, where $0 \le x^* \le 1$.
    \item Randomized rounding - Pick set $i$ with probability $X_i^* \cdot 2 H_n$.
\end{enumerate}

\begin{claim}
    The sets picked form a set cover with high probability.
\end{claim}
\begin{proof}
    Consider an element $e_j$. It lies in some sets, say $i_1, \ldots, i_k$. If any of the sets has $x_i^* \ge \frac{1}{2 H_n}$, we are done. Otherwise:\nl
    \begin{align*}
        P(e_i \text{ is not covered}) &= \prod_i (1 - 2 x_i^* H_n)\\
                              &\le \prod_i e^{-2 x_i^* H_n}\\
                              &= e^{\sum_{i : e_j \in S_i} 2 x_i^* H_n}\\
                              &\le e^{-2H_n}\\
                              &\le \frac{1}{n^2}
    \end{align*}
    So the probability that some element is not covered is $\le \frac{1}{n}$ by the union bound. Hence the probability that this is a set cover is $\ge 1 - \frac{1}{n}$.
\end{proof}

Number of sets picked is

\begin{align*}
    \mathbb{E}\left[\textbf{1}_{S_i}\right] &= \sum_i 2 x_i^* H_n\\
                       &\le 2 H_n \cdot \OPT
\end{align*}

\end{document}
