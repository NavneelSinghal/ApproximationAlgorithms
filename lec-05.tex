\documentclass[a4paper]{article}
\usepackage[english]{babel}
\usepackage[a4paper,top=2cm,bottom=2cm,left=2cm,right=2cm,marginparwidth=1.75cm]{geometry}
\usepackage{amsmath}
\usepackage{amsfonts}
% \usepackage{amsthm}
\usepackage{amssymb}
\usepackage{graphicx}
\usepackage[colorinlistoftodos]{todonotes}
\usepackage[colorlinks=true, allcolors=blue]{hyperref}
\usepackage{import}
\usepackage{pdfpages}
\usepackage{transparent}
\usepackage{xcolor}
\usepackage{algorithmicx}
\usepackage{algpseudocode}

\usepackage{thmtools}
\usepackage{enumitem}
\usepackage[framemethod=TikZ]{mdframed}

\usepackage{xpatch}

\usepackage{boites}
\makeatletter
\xpatchcmd{\endmdframed}
{\aftergroup\endmdf@trivlist\color@endgroup}
{\endmdf@trivlist\color@endgroup\@doendpe}
{}{}
\makeatother

%\usepackage[poster]{tcolorbox}
%\allowdisplaybreaks
%\sloppy

\usepackage[many]{tcolorbox}

\xpatchcmd{\proof}{\itshape}{\bfseries\itshape}{}{}

% to set box separation
\setlength{\fboxsep}{0.8em}
\def\breakboxskip{7pt}
\def\breakboxparindent{0em}

\newenvironment{proof}{\begin{breakbox}\textit{Proof.}}{\hfill$\square$\end{breakbox}}
\newenvironment{ans}{\begin{breakbox}\textit{Answer.}}{\end{breakbox}}
\newenvironment{soln}{\begin{breakbox}\textit{Solution.}}{\end{breakbox}}

% \tcolorboxenvironment{proof}{
%     blanker,
%     before skip=\topsep,
%     after skip=\topsep,
%     borderline={0.4pt}{0.4pt}{black},
%     breakable,
%     left=12pt,
%     right=12pt,
%     top=12pt,
%     bottom=12pt,
% }
%
% \tcolorboxenvironment{ans}{
%     blanker,
%     before skip=\topsep,
%     after skip=\topsep,
%     borderline={0.4pt}{0.4pt}{black},
%     breakable,
%     left=12pt,
%     right=12pt,
% }

\mdfdefinestyle{enclosed}{
    linecolor=black
    ,backgroundcolor=none
    ,apptotikzsetting={\tikzset{mdfbackground/.append style={fill=gray!100,fill opacity=.3}}}
    ,frametitlefont=\sffamily\bfseries\color{black}
    ,splittopskip=.5cm
    ,frametitlebelowskip=.0cm
    ,topline=true
    ,bottomline=true
    ,rightline=true
    ,leftline=true
    ,leftmargin=0.01cm
    ,linewidth=0.02cm
    ,skipabove=0.01cm
    ,innerbottommargin=0.1cm
    ,skipbelow=0.1cm
}

\mdfsetup{%
    middlelinecolor=black,
    middlelinewidth=1pt,
roundcorner=4pt}

\setlength{\parindent}{0pt}

\mdtheorem[style=enclosed]{theorem}{Theorem}
\mdtheorem[style=enclosed]{lemma}{Lemma}[theorem]
\mdtheorem[style=enclosed]{claim}{Claim}[theorem]
\mdtheorem[style=enclosed]{ques}{Question}
\mdtheorem[style=enclosed]{defn}{Definition}
\mdtheorem[style=enclosed]{notn}{Notation}
\mdtheorem[style=enclosed]{obs}{Observation}
\mdtheorem[style=enclosed]{eg}{Example}
\mdtheorem[style=enclosed]{cor}{Corollary}
\mdtheorem[style=enclosed]{note}{Note}

% \let\thetheorem=\relax
% \let\thelemma=\relax
% \let\theclaim=\relax
% \let\theques=\relax
% \let\thedefn=\relax
% \let\thenotn=\relax
% \let\theobs=\relax
% \let\thecor=\relax
% \let\thenote=\relax

% \renewcommand\qedsymbol{$\blacksquare$}
\newcommand{\nl}{\vspace{0.2cm}\\}
\newcommand{\ol}{\overline}
\newcommand{\mc}{\mathcal}
\newcommand{\mi}{\mathit}
\newcommand{\mf}{\mathbf}
\newcommand{\mb}{\mathbb}
\newcommand{\OPT}{\mathbf{OPT}}
\newcommand{\ALG}{\mathbf{ALG}}
\renewcommand{\L}{\mc{L}}
\newcommand{\changesto}{\vdash}
\newcommand\Vtextvisiblespace[1][.3em]{%
    \mbox{\kern.06em\vrule height.3ex}%
    \vbox{\hrule width#1}%
    \hbox{\vrule height.3ex}
}
\newcommand{\blank}{{\Vtextvisiblespace[0.7em]}}
\newcommand{\leftend}{\triangleright}
\newcommand{\comp}{\overline}

\newcommand{\incfig}[1]{%
    \def\svgwidth{\columnwidth}
    \import{./figures/}{#1.pdf_tex}
}
\pdfsuppresswarningpagegroup=1

\title{\textbf{Approximation Algorithms Lecture 5}}
\date{}

%\section{Recap}
%
%\section{Definitions}
%
%\begin{defn}
%\end{defn}
%
%\section{Content}
%
%\begin{theorem}
%\end{theorem}
%\begin{proof}
%\end{proof}
%
%\begin{ques}
%\end{ques}
%
%\begin{eg}
%\end{eg}
%
%\begin{claim}
%\end{claim}

\begin{document}
\maketitle
\tableofcontents

\section{Recap}

\subsection{Vertex Cover}

Given an undirected unweighted graph $G = (V, E)$, a vertex cover is a set $S \subseteq V$ such that each edge $e$ in $E$ has at least one endpoint in $S$. Find the minimum vertex cover in a given
graph $G$.\nl

\textbf{Matching}\nl
A matching in a graph $G = (V, E)$ is a subset of edges which are independent (two edges are independent if they don't share an endpoint).\nl

\textbf{Maximal matching}\nl
A matching is \underline{maximal} if its size can't be increasing by adding another edge.\nl

\begin{claim}
    Minimum vertex cover is at least as large as any matching.
\end{claim}
\begin{proof}
    Note that for any edge in the matching, at least one vertex is needed in the vertex cover. Hence the size of any vertex cover is at least the size of any matching.
\end{proof}

\textbf{Algorithm}
\begin{enumerate}
    \item Find a maximal matching in $G$.
    \item For every matched edge, put both endpoints of the edge in the solution $S$.
\end{enumerate}

\begin{claim}
    $S$ is a vertex cover.
\end{claim}

\begin{proof}
    Suppose there is an edge $e$ which is not adjacent to any vertex in $S$. We claim that we can add this edge to the matching, which will contradict the maximality. Note that $S$ is the set of vertices
    which are adjacent to at least one edge in the matching. Since $e$ is not incident on any vertex in $S$, there is no edge in the matching that shares an endpoint with $e$, whence we are done.
\end{proof}

\begin{claim}
    $|S| \le 2 |\OPT|$.
\end{claim}
\begin{proof}
    Note that $|\OPT|$ is the size of the minimum vertex cover. Then we have $|\OPT| \ge |M|$. But we have $|S| \le 2|M|$, which gives us $|S| \le 2|M| \le 2|\OPT|$, as needed. 
\end{proof}

\section{Content}

\subsection{Steiner Tree}

Suppose you're given a graph $G = (V, E)$, and a subset $T \subseteq V$ called the set of terminals, and a function $\ell : E \to \mathbb{R}^+$ corresponding to edge length. Find a tree which includes all terminals and has
minimum length.\nl

\textbf{Greedy Algorithm}\nl
Construct a new graph $G' = (T, E' = T \times T)$, and $\ell' = T \times T \to \mathbb{R}^+$, where $\ell'(u, v)$ is the length of the shortest path between $u$ and $v$ in $G$. Now
corresponding to the minimum spanning tree in $G'$, replace each edge of this MST with the corresponding path in $G$, and find the MST of this newly formed connected subgraph.\nl
\begin{claim}
    The solution obtained is at most twice the optimum.
\end{claim}
\begin{proof}
    Consider the optimal Steiner tree. Note that due to optimality, the leaves have to be terminals (not all terminals are necessarily leaves). Consider the Euler traversal of this tree.\nl
    Note that every edge appears twice in this Euler traversal, and that the Euler traversal corresponds to a cycle in $G'$ (which includes all terminals) which can be formed by going to the next
    vertex which is not visited.\nl
    Hence the length of this cycle is at most $2 \cdot \OPT$. And the MST of the graph $G'$ has length at most the length of any cycle through all edges in the graph times $1 - \frac{1}{k}$ (after
    dropping the heaviest edge in this cycle). So the MST of
    $G'$ has length at most $2\left(1 - \frac{1}{k}\right) \cdot \OPT$.
\end{proof}

\end{document}
