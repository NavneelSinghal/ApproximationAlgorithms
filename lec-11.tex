\documentclass[a4paper]{article}
\usepackage[english]{babel}
\usepackage[a4paper,top=2cm,bottom=2cm,left=2cm,right=2cm,marginparwidth=1.75cm]{geometry}
\usepackage{amsmath}
\usepackage{amsfonts}
% \usepackage{amsthm}
\usepackage{amssymb}
\usepackage{graphicx}
\usepackage[colorinlistoftodos]{todonotes}
\usepackage[colorlinks=true, allcolors=blue]{hyperref}
\usepackage{import}
\usepackage{pdfpages}
\usepackage{transparent}
\usepackage{xcolor}
\usepackage{algorithmicx}
\usepackage{algpseudocode}

\usepackage{thmtools}
\usepackage{enumitem}
\usepackage[framemethod=TikZ]{mdframed}

\usepackage{xpatch}

\usepackage{boites}
\makeatletter
\xpatchcmd{\endmdframed}
{\aftergroup\endmdf@trivlist\color@endgroup}
{\endmdf@trivlist\color@endgroup\@doendpe}
{}{}
\makeatother

%\usepackage[poster]{tcolorbox}
%\allowdisplaybreaks
%\sloppy

\usepackage[many]{tcolorbox}

\xpatchcmd{\proof}{\itshape}{\bfseries\itshape}{}{}

% to set box separation
\setlength{\fboxsep}{0.8em}
\def\breakboxskip{7pt}
\def\breakboxparindent{0em}

\newenvironment{proof}{\begin{breakbox}\textit{Proof.}}{\hfill$\square$\end{breakbox}}
\newenvironment{ans}{\begin{breakbox}\textit{Answer.}}{\end{breakbox}}
\newenvironment{soln}{\begin{breakbox}\textit{Solution.}}{\end{breakbox}}

% \tcolorboxenvironment{proof}{
%     blanker,
%     before skip=\topsep,
%     after skip=\topsep,
%     borderline={0.4pt}{0.4pt}{black},
%     breakable,
%     left=12pt,
%     right=12pt,
%     top=12pt,
%     bottom=12pt,
% }
%
% \tcolorboxenvironment{ans}{
%     blanker,
%     before skip=\topsep,
%     after skip=\topsep,
%     borderline={0.4pt}{0.4pt}{black},
%     breakable,
%     left=12pt,
%     right=12pt,
% }

\mdfdefinestyle{enclosed}{
    linecolor=black
    ,backgroundcolor=none
    ,apptotikzsetting={\tikzset{mdfbackground/.append style={fill=gray!100,fill opacity=.3}}}
    ,frametitlefont=\sffamily\bfseries\color{black}
    ,splittopskip=.5cm
    ,frametitlebelowskip=.0cm
    ,topline=true
    ,bottomline=true
    ,rightline=true
    ,leftline=true
    ,leftmargin=0.01cm
    ,linewidth=0.02cm
    ,skipabove=0.01cm
    ,innerbottommargin=0.1cm
    ,skipbelow=0.1cm
}

\mdfsetup{%
    middlelinecolor=black,
    middlelinewidth=1pt,
roundcorner=4pt}

\setlength{\parindent}{0pt}

\mdtheorem[style=enclosed]{theorem}{Theorem}
\mdtheorem[style=enclosed]{lemma}{Lemma}[theorem]
\mdtheorem[style=enclosed]{claim}{Claim}[theorem]
\mdtheorem[style=enclosed]{ques}{Question}
\mdtheorem[style=enclosed]{defn}{Definition}
\mdtheorem[style=enclosed]{notn}{Notation}
\mdtheorem[style=enclosed]{obs}{Observation}
\mdtheorem[style=enclosed]{eg}{Example}
\mdtheorem[style=enclosed]{cor}{Corollary}
\mdtheorem[style=enclosed]{note}{Note}

% \let\thetheorem=\relax
% \let\thelemma=\relax
% \let\theclaim=\relax
% \let\theques=\relax
% \let\thedefn=\relax
% \let\thenotn=\relax
% \let\theobs=\relax
% \let\thecor=\relax
% \let\thenote=\relax

% \renewcommand\qedsymbol{$\blacksquare$}
\newcommand{\nl}{\vspace{0.2cm}\\}
\newcommand{\ol}{\overline}
\newcommand{\eps}{\varepsilon}
\newcommand{\mc}{\mathcal}
\newcommand{\mi}{\mathit}
\newcommand{\mf}{\mathbf}
\newcommand{\mb}{\mathbb}
\newcommand{\R}{\mathbb{R}}
\newcommand{\OPT}{\mathbf{OPT}}
\newcommand{\ALG}{\mathbf{ALG}}
\renewcommand{\L}{\mc{L}}
\newcommand{\changesto}{\vdash}
\newcommand\Vtextvisiblespace[1][.3em]{%
    \mbox{\kern.06em\vrule height.3ex}%
    \vbox{\hrule width#1}%
    \hbox{\vrule height.3ex}
}
\newcommand{\blank}{{\Vtextvisiblespace[0.7em]}}
\newcommand{\leftend}{\triangleright}
\newcommand{\comp}{\overline}

\newcommand{\incfig}[1]{%
    \def\svgwidth{\columnwidth}
    \import{./figures/}{#1.pdf_tex}
}
\pdfsuppresswarningpagegroup=1

\title{\textbf{Approximation Algorithms Lecture 11}}
\date{}

%\section{Recap}
%
%\section{Definitions}
%
%\begin{defn}
%\end{defn}
%
%\section{Content}
%
%\begin{theorem}
%\end{theorem}
%\begin{proof}
%\end{proof}
%
%\begin{ques}
%\end{ques}
%
%\begin{eg}
%\end{eg}
%
%\begin{claim}
%\end{claim}

\begin{document}
\maketitle
\tableofcontents

\section{Recap}

Load balancing PTAS. Read the book to get the correct proof.

\section{Content}

\subsection{Fully polynomial time approximation scheme (FPTAS) for knapsack problem}

There is a knapsack. You're given $n$ objects, the $i^{th}$ object has weight $w_i$ and value $p_i$, both in $\R^+$.\nl
The knapsack can carry a weight of $W$. Pick a subset of objects of weight at most $w$ and having maximum value.\nl
Suppose the weights are real numbers and the values are positive integers.\nl
Dynamic programming: $dp(i, j)$ is the minimum weight of a subset of $\{1, \ldots, i\}$ which has a profit of $j$.\nl
Then we have $dp(i, j) = \min \{dp(i - 1, j), w_i + dp(i - 1, j - p_i)\}$. The time required is $O(nP)$, which is not polynomial time in the input, since the size of input is $O(n \log W_{max} + n \log
P_{max})$, and $P$ is the total profit, which is at least $P_{max}$.\nl
Now our aim is to get a polynomial time approximation. So we will build a $(1 + \eps)$ approximation that has running time polynomial in the input length and $\frac{1}{\eps}$.\nl
Idea: Scale profits by $k$, which will be determined later. Then $p_i' = \lfloor \frac{p_i}{k} \rfloor$. So with the scaled profits, the running time becomes $O(n P/k)$.\nl
How about the profit of the solution obtained by this method? How does it compare to $\OPT$?\nl
Let $I'$ be the set of objects picked by this algorithm, with profit function $p'$, and let $I$ be the set of objects picked in the optimal solution, with profit function $p$.\nl
We have $profit(I') = \sum_{i \in I'} p_i \ge k \sum_{i \in I'} p_i' \ge k \sum_{i \in I} p_i' > k \sum_{i \in I} \frac{p_i}{k} - 1 = \sum_{i \in I} p_i - k |I| = \OPT - k|I| \ge \OPT - kn$.\nl
So we want $kn < \eps \OPT$, i.e., $k \le \frac{\eps \OPT}{n}$.\nl
Set $k = \frac{\eps P}{n}$. Note that $P \le \OPT$. \todo{This inequality is probably reversed}\nl
The running time is polynomial in $n, 1/\eps$. These are called fully polynomial time approximation schemes.\nl
\begin{theorem}
    There can be no PTAS for bin packing.
\end{theorem}
\begin{proof}
    Given a bin-packing instance, it is NP-hard to determine if the objects can be packed in two bins. This is by showing a reduction to the subset sum problem.\nl
    Now suppose we have a PTAS for bin packing. Then we claim that we can solve this problem.\nl
    Consider the PTAS for bin packing. Find an approximation of $\eps < 1/2$ using the PTAS. If the PTAS for bin packing has $2$, then yes, else no. This solves the subset sum problem.
\end{proof}

We will next look at asymptotic PTAS.

\end{document}
