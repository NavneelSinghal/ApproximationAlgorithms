\documentclass[a4paper]{article}
\usepackage[english]{babel}
\usepackage[a4paper,top=2cm,bottom=2cm,left=2cm,right=2cm,marginparwidth=1.75cm]{geometry}
\usepackage{amsmath}
\usepackage{amsfonts}
% \usepackage{amsthm}
\usepackage{amssymb}
\usepackage{graphicx}
\usepackage[colorinlistoftodos]{todonotes}
\usepackage[colorlinks=true, allcolors=blue]{hyperref}
\usepackage{import}
\usepackage{pdfpages}
\usepackage{transparent}
\usepackage{xcolor}
\usepackage{algorithmicx}
\usepackage{algpseudocode}

\usepackage{thmtools}
\usepackage{enumitem}
\usepackage[framemethod=TikZ]{mdframed}

\usepackage{xpatch}

\usepackage{boites}
\makeatletter
\xpatchcmd{\endmdframed}
{\aftergroup\endmdf@trivlist\color@endgroup}
{\endmdf@trivlist\color@endgroup\@doendpe}
{}{}
\makeatother

%\usepackage[poster]{tcolorbox}
%\allowdisplaybreaks
%\sloppy

\usepackage[many]{tcolorbox}

\xpatchcmd{\proof}{\itshape}{\bfseries\itshape}{}{}

% to set box separation
\setlength{\fboxsep}{0.8em}
\def\breakboxskip{7pt}
\def\breakboxparindent{0em}

\newenvironment{proof}{\begin{breakbox}\textit{Proof.}}{\hfill$\square$\end{breakbox}}
\newenvironment{ans}{\begin{breakbox}\textit{Answer.}}{\end{breakbox}}
\newenvironment{soln}{\begin{breakbox}\textit{Solution.}}{\end{breakbox}}

% \tcolorboxenvironment{proof}{
%     blanker,
%     before skip=\topsep,
%     after skip=\topsep,
%     borderline={0.4pt}{0.4pt}{black},
%     breakable,
%     left=12pt,
%     right=12pt,
%     top=12pt,
%     bottom=12pt,
% }
%
% \tcolorboxenvironment{ans}{
%     blanker,
%     before skip=\topsep,
%     after skip=\topsep,
%     borderline={0.4pt}{0.4pt}{black},
%     breakable,
%     left=12pt,
%     right=12pt,
% }

\mdfdefinestyle{enclosed}{
    linecolor=black
    ,backgroundcolor=none
    ,apptotikzsetting={\tikzset{mdfbackground/.append style={fill=gray!100,fill opacity=.3}}}
    ,frametitlefont=\sffamily\bfseries\color{black}
    ,splittopskip=.5cm
    ,frametitlebelowskip=.0cm
    ,topline=true
    ,bottomline=true
    ,rightline=true
    ,leftline=true
    ,leftmargin=0.01cm
    ,linewidth=0.02cm
    ,skipabove=0.01cm
    ,innerbottommargin=0.1cm
    ,skipbelow=0.1cm
}

\mdfsetup{%
    middlelinecolor=black,
    middlelinewidth=1pt,
roundcorner=4pt}

\setlength{\parindent}{0pt}

\mdtheorem[style=enclosed]{theorem}{Theorem}
\mdtheorem[style=enclosed]{lemma}{Lemma}[theorem]
\mdtheorem[style=enclosed]{claim}{Claim}[theorem]
\mdtheorem[style=enclosed]{ques}{Question}
\mdtheorem[style=enclosed]{defn}{Definition}
\mdtheorem[style=enclosed]{notn}{Notation}
\mdtheorem[style=enclosed]{obs}{Observation}
\mdtheorem[style=enclosed]{eg}{Example}
\mdtheorem[style=enclosed]{cor}{Corollary}
\mdtheorem[style=enclosed]{note}{Note}

% \let\thetheorem=\relax
% \let\thelemma=\relax
% \let\theclaim=\relax
% \let\theques=\relax
% \let\thedefn=\relax
% \let\thenotn=\relax
% \let\theobs=\relax
% \let\thecor=\relax
% \let\thenote=\relax

% \renewcommand\qedsymbol{$\blacksquare$}
\newcommand{\nl}{\vspace{0.2cm}\\}
\newcommand{\ol}{\overline}
\newcommand{\eps}{\varepsilon}
\newcommand{\mc}{\mathcal}
\newcommand{\mi}{\mathit}
\newcommand{\mf}{\mathbf}
\newcommand{\mb}{\mathbb}
\newcommand{\R}{\mathbb{R}}
\newcommand{\Z}{\mathbb{Z}}
\newcommand{\OPT}{\mathbf{OPT}}
\newcommand{\ALG}{\mathbf{ALG}}
\renewcommand{\L}{\mc{L}}
\newcommand{\changesto}{\vdash}
\newcommand\Vtextvisiblespace[1][.3em]{%
    \mbox{\kern.06em\vrule height.3ex}%
    \vbox{\hrule width#1}%
    \hbox{\vrule height.3ex}
}
\newcommand{\blank}{{\Vtextvisiblespace[0.7em]}}
\newcommand{\leftend}{\triangleright}
\newcommand{\comp}{\overline}

\newcommand{\incfig}[1]{%
    \def\svgwidth{\columnwidth}
    \import{./figures/}{#1.pdf_tex}
}
\pdfsuppresswarningpagegroup=1

\title{\textbf{Approximation Algorithms Lecture 24}}
\date{}

%\section{Recap}
%
%\section{Definitions}
%
%\begin{defn}
%\end{defn}
%
%\section{Content}
%
%\begin{theorem}
%\end{theorem}
%\begin{proof}
%\end{proof}
%
%\begin{ques}
%\end{ques}
%
%\begin{eg}
%\end{eg}
%
%\begin{claim}
%\end{claim}

\begin{document}
\maketitle
\tableofcontents

\section{Recap}

Completion of primal dual algorithms for steiner forest.

\section{Content}

\subsection{Maximum cut in a graph}
\begin{defn}
    \textbf{Cut:} Given an undirected and unweighted graph $G = (V, E)$, a \emph{cut} is a partition of the vertex set into two sets $S$ and $V \setminus S$. The edges of the cut are the edges whose endpoints are in different parts of this partition.\nl
The size/weight of the cut is the number (or total weight) of edges in the cut.
\end{defn}
The maximum cut problem is to find a cut of the maximum size.\nl
In a bipartite graph, the number of edges is the actual max cut. For a general graph, the max cut is at most the number of edges in the graph.\nl
\textbf{Local search algorithm for max cut}
\begin{enumerate}
    \item Take an arbitrary partition of the vertex set $S, V \setminus S$.
        \begin{itemize}
            \item If there exists a vertex such that moving it to the other partition increases the size of the cut, then we make the change.
            \item Repeat this until no such change is possible.
        \end{itemize}
    \item Return the final cut $(S, V \setminus S)$.
\end{enumerate}
The operation is also called a \emph{local move}, and the final solution is called a locally optimal solution.\nl
\begin{claim}
This algorithm returns a cut of size at least half the number of edges in the graph which is at least $\frac{\OPT}{2}$.
\end{claim}
\begin{proof}
Consider a locally optimum solution $(S^*, V \setminus S^*)$. Consider any vertex $v \in S^*$, then we have $\deg_{S^*}(v) \le \deg_{V \setminus S^*}(v)$. This is true for all vertices in $S^*$.
Summing for all vertices in $S^*$, we get the LHS as $2$ times the number of edges in $S^*$, and the RHS as number of edges in the cut $(S^*, V \setminus S^*)$. By summing the corresponding inequality
for all vertices in $V \setminus S^*$, we get that the number of edges in the cut is at lesat twice the number of edges in $V \setminus S^*$. Now adding twice the number of edges in $V \setminus
S^*$ to the inequality found by adding these two, we get $2|E| \le 4$ times the number of edges in the cut, and we are done.
\end{proof}
This can be implemented in $O(m \cdot (n + \max \deg(v)))$.\nl

\subsection{Balanced cuts}
A cut which has equal number of vertices in each set of the partition.\nl
A relaxed version is where no part has number of vertices more than twice that of another part. For cuts, this implies that each part has at least 1/3rd and at most 2/3rd the vertices of the
graph.\nl
Suppose we want to find a balanced max-cut (exactly half the vertices on each side). How to modify the previous algorithm?
\begin{enumerate}
    \item Pick an arbitrary balanced cut $(S, V \setminus S)$.
        \begin{itemize}
            \item If $\exists u \in S, v \in V \setminus S$ such that swapping the pair increases the size of the cut, then we do so.
            \item Repeat while such a pair exists.
        \end{itemize}
    \item Return the locally optimal solution found.
\end{enumerate}
Let's analyse this algorithm.\nl
Let green edges be edges across sets starting from either of $u, v$ but not both, and let blue ones be the one within the sets starting from either of $u, v$.
When will we make the swap? We make the swap if the number of blue edges is greater than the number of green edges.\nl
Suppose $(S^* = S, V \setminus S^*)$ is a locally optimal solution. We pair vertices in $S^*$ with the vertices in $V \setminus S^* = T$ in an arbitrary manner.\nl
Then it holds that $\deg_T (u_i) + \deg_S (v_i) \ge \deg_S(u_i) + \deg_T(v_i)$ where we also include the red edges in the LHS.\nl
Summing this inequality, we get $2 \cdot (\text{number of edges in both sets}) \le 2 \cdot (\text{number of edges in the cut})$.\nl
Hence we get the same conclusion as before.\nl
So we get a $\frac{1}{2}$ approximation algorithm for balanced max-cut.

\end{document}
