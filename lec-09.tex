\documentclass[a4paper]{article}
\usepackage[english]{babel}
\usepackage[a4paper,top=2cm,bottom=2cm,left=2cm,right=2cm,marginparwidth=1.75cm]{geometry}
\usepackage{amsmath}
\usepackage{amsfonts}
% \usepackage{amsthm}
\usepackage{amssymb}
\usepackage{graphicx}
\usepackage[colorinlistoftodos]{todonotes}
\usepackage[colorlinks=true, allcolors=blue]{hyperref}
\usepackage{import}
\usepackage{pdfpages}
\usepackage{transparent}
\usepackage{xcolor}
\usepackage{algorithmicx}
\usepackage{algpseudocode}

\usepackage{thmtools}
\usepackage{enumitem}
\usepackage[framemethod=TikZ]{mdframed}

\usepackage{xpatch}

\usepackage{boites}
\makeatletter
\xpatchcmd{\endmdframed}
{\aftergroup\endmdf@trivlist\color@endgroup}
{\endmdf@trivlist\color@endgroup\@doendpe}
{}{}
\makeatother

%\usepackage[poster]{tcolorbox}
%\allowdisplaybreaks
%\sloppy

\usepackage[many]{tcolorbox}

\xpatchcmd{\proof}{\itshape}{\bfseries\itshape}{}{}

% to set box separation
\setlength{\fboxsep}{0.8em}
\def\breakboxskip{7pt}
\def\breakboxparindent{0em}

\newenvironment{proof}{\begin{breakbox}\textit{Proof.}}{\hfill$\square$\end{breakbox}}
\newenvironment{ans}{\begin{breakbox}\textit{Answer.}}{\end{breakbox}}
\newenvironment{soln}{\begin{breakbox}\textit{Solution.}}{\end{breakbox}}

% \tcolorboxenvironment{proof}{
%     blanker,
%     before skip=\topsep,
%     after skip=\topsep,
%     borderline={0.4pt}{0.4pt}{black},
%     breakable,
%     left=12pt,
%     right=12pt,
%     top=12pt,
%     bottom=12pt,
% }
%
% \tcolorboxenvironment{ans}{
%     blanker,
%     before skip=\topsep,
%     after skip=\topsep,
%     borderline={0.4pt}{0.4pt}{black},
%     breakable,
%     left=12pt,
%     right=12pt,
% }

\mdfdefinestyle{enclosed}{
    linecolor=black
    ,backgroundcolor=none
    ,apptotikzsetting={\tikzset{mdfbackground/.append style={fill=gray!100,fill opacity=.3}}}
    ,frametitlefont=\sffamily\bfseries\color{black}
    ,splittopskip=.5cm
    ,frametitlebelowskip=.0cm
    ,topline=true
    ,bottomline=true
    ,rightline=true
    ,leftline=true
    ,leftmargin=0.01cm
    ,linewidth=0.02cm
    ,skipabove=0.01cm
    ,innerbottommargin=0.1cm
    ,skipbelow=0.1cm
}

\mdfsetup{%
    middlelinecolor=black,
    middlelinewidth=1pt,
roundcorner=4pt}

\setlength{\parindent}{0pt}

\mdtheorem[style=enclosed]{theorem}{Theorem}
\mdtheorem[style=enclosed]{lemma}{Lemma}[theorem]
\mdtheorem[style=enclosed]{claim}{Claim}[theorem]
\mdtheorem[style=enclosed]{ques}{Question}
\mdtheorem[style=enclosed]{defn}{Definition}
\mdtheorem[style=enclosed]{notn}{Notation}
\mdtheorem[style=enclosed]{obs}{Observation}
\mdtheorem[style=enclosed]{eg}{Example}
\mdtheorem[style=enclosed]{cor}{Corollary}
\mdtheorem[style=enclosed]{note}{Note}

% \let\thetheorem=\relax
% \let\thelemma=\relax
% \let\theclaim=\relax
% \let\theques=\relax
% \let\thedefn=\relax
% \let\thenotn=\relax
% \let\theobs=\relax
% \let\thecor=\relax
% \let\thenote=\relax

% \renewcommand\qedsymbol{$\blacksquare$}
\newcommand{\nl}{\vspace{0.2cm}\\}
\newcommand{\ol}{\overline}
\newcommand{\eps}{\varepsilon}
\newcommand{\mc}{\mathcal}
\newcommand{\mi}{\mathit}
\newcommand{\mf}{\mathbf}
\newcommand{\mb}{\mathbb}
\newcommand{\R}{\mathbb{R}}
\newcommand{\OPT}{\mathbf{OPT}}
\newcommand{\ALG}{\mathbf{ALG}}
\renewcommand{\L}{\mc{L}}
\newcommand{\changesto}{\vdash}
\newcommand\Vtextvisiblespace[1][.3em]{%
    \mbox{\kern.06em\vrule height.3ex}%
    \vbox{\hrule width#1}%
    \hbox{\vrule height.3ex}
}
\newcommand{\blank}{{\Vtextvisiblespace[0.7em]}}
\newcommand{\leftend}{\triangleright}
\newcommand{\comp}{\overline}

\newcommand{\incfig}[1]{%
    \def\svgwidth{\columnwidth}
    \import{./figures/}{#1.pdf_tex}
}
\pdfsuppresswarningpagegroup=1

\title{\textbf{Approximation Algorithms Lecture 9}}
\date{}

%\section{Recap}
%
%\section{Definitions}
%
%\begin{defn}
%\end{defn}
%
%\section{Content}
%
%\begin{theorem}
%\end{theorem}
%\begin{proof}
%\end{proof}
%
%\begin{ques}
%\end{ques}
%
%\begin{eg}
%\end{eg}
%
%\begin{claim}
%\end{claim}

\begin{document}
\maketitle
\tableofcontents

\section{Recap}

$k$-center problem.

\section{Content}

\subsection{Load balancing revisited}

Let's come back to load balancing. PTAS = polynomial time approximation scheme. We will try to do some sort of packing.

\subsection{Bin packing}
Given $n$ objects of sizes $0 \le s_1, \ldots, s_n \le 1$. What is the minimum number of bins of size $1$ needed to pack the objects.\nl
Bin packing can be solved optimally in $O(n^{2s})$ time, where $s$
is the number of distinct sizes.\nl
\begin{proof}
    Let $n_1$ be the number of objects of size $s_1$. An instance of bin packing is a vector $(n_1, n_2, \cdots, n_s)$ and the sizes $s_1, \cdots, s_s$.\nl
    Let $B(o_1, \cdots, o_s)$ be the number of bins needed to pack $o_i$ objects of size $s_i$. We are interested in $B(n_1, \cdots, n_s)$.\nl
    $$B(o_1, o_2, \cdots, o_s) = \min_{c_1, \cdots, c_s \ge 0, \sum_{i=1}^s c_i s_i \le 1} B(o_1 - c_1, \cdots, o_s - c_s) + 1$$
    with $B(0, 0, \cdots, 0) = 0$.\nl
    Let $B$ be an $s-$dimensional matrix whose $(o_1, \cdots, o_s)$-th entry is $B(o_1, \cdots, o_s)$. This has $O(n^s)$ entries. To compute an entry, we would have to check all possible choices of
    $(c_1, \cdots, c_s)$ where $c_i \le o_i$. Total number of choices is $\prod_{i=1}^s (1 + o_i) \le (n + 1)^s$.
    Total time required to fill this table is thus $O(n^{2s})$.
\end{proof}

Is there a way of assigning jobs to machines such that the makespan is at most $T$? TO check this, we do $s_i = p_i / T$, and if the number of bins is $\le m$, then the answer is yes, else it
is no. Now do a binary search on $T$ to solve for the final answer.\nl
This gives an $O(n^{2s} \cdot T(\text{binary search}))$ time algorithm for load balancing where $s$ is the number of different processing times.\nl

\textbf{Rounding processing times}\nl
\begin{enumerate}
    \item Ignore all jobs with processign times less than $\epsilon T$.
    \item Round down processing times to the nearest value to $\eps T (1 + \eps)^k$.
\end{enumerate}

Note that $\eps (1 + \eps)^{k-1} = 1$, so $k = \lceil\log_{1 + \eps} \frac{1}{\eps} \rceil$.\nl
Given a bin-packing solution that uses $\le m$ bins, we can obtain a solution for load balancing that has makespan at most $T(1 + \eps)$. For the jobs that we didn't ignore, we get this trivially,
since the weight just after the weight we rounded it down to was more than the original size. So we get a makespan of $T(1 + \eps)$. It might be possible that the small jobs can possibly exceed the
whole thing. We'll do it in the next class.

\end{document}
