\documentclass[a4paper]{article}
\usepackage[english]{babel}
\usepackage[a4paper,top=2cm,bottom=2cm,left=2cm,right=2cm,marginparwidth=1.75cm]{geometry}
\usepackage{amsmath}
\usepackage{amsfonts}
% \usepackage{amsthm}
\usepackage{amssymb}
\usepackage{graphicx}
\usepackage[colorinlistoftodos]{todonotes}
\usepackage[colorlinks=true, allcolors=blue]{hyperref}
\usepackage{import}
\usepackage{pdfpages}
\usepackage{transparent}
\usepackage{xcolor}
\usepackage{algorithmicx}
\usepackage{algpseudocode}

\usepackage{thmtools}
\usepackage{enumitem}
\usepackage[framemethod=TikZ]{mdframed}

\usepackage{xpatch}

\usepackage{boites}
\makeatletter
\xpatchcmd{\endmdframed}
{\aftergroup\endmdf@trivlist\color@endgroup}
{\endmdf@trivlist\color@endgroup\@doendpe}
{}{}
\makeatother

%\usepackage[poster]{tcolorbox}
%\allowdisplaybreaks
%\sloppy

\usepackage[many]{tcolorbox}

\xpatchcmd{\proof}{\itshape}{\bfseries\itshape}{}{}

% to set box separation
\setlength{\fboxsep}{0.8em}
\def\breakboxskip{7pt}
\def\breakboxparindent{0em}

\newenvironment{proof}{\begin{breakbox}\textit{Proof.}}{\hfill$\square$\end{breakbox}}
\newenvironment{ans}{\begin{breakbox}\textit{Answer.}}{\end{breakbox}}
\newenvironment{soln}{\begin{breakbox}\textit{Solution.}}{\end{breakbox}}

% \tcolorboxenvironment{proof}{
%     blanker,
%     before skip=\topsep,
%     after skip=\topsep,
%     borderline={0.4pt}{0.4pt}{black},
%     breakable,
%     left=12pt,
%     right=12pt,
%     top=12pt,
%     bottom=12pt,
% }
%
% \tcolorboxenvironment{ans}{
%     blanker,
%     before skip=\topsep,
%     after skip=\topsep,
%     borderline={0.4pt}{0.4pt}{black},
%     breakable,
%     left=12pt,
%     right=12pt,
% }

\mdfdefinestyle{enclosed}{
    linecolor=black
    ,backgroundcolor=none
    ,apptotikzsetting={\tikzset{mdfbackground/.append style={fill=gray!100,fill opacity=.3}}}
    ,frametitlefont=\sffamily\bfseries\color{black}
    ,splittopskip=.5cm
    ,frametitlebelowskip=.0cm
    ,topline=true
    ,bottomline=true
    ,rightline=true
    ,leftline=true
    ,leftmargin=0.01cm
    ,linewidth=0.02cm
    ,skipabove=0.01cm
    ,innerbottommargin=0.1cm
    ,skipbelow=0.1cm
}

\mdfsetup{%
    middlelinecolor=black,
    middlelinewidth=1pt,
roundcorner=4pt}

\setlength{\parindent}{0pt}

\mdtheorem[style=enclosed]{theorem}{Theorem}
\mdtheorem[style=enclosed]{lemma}{Lemma}[theorem]
\mdtheorem[style=enclosed]{claim}{Claim}[theorem]
\mdtheorem[style=enclosed]{ques}{Question}
\mdtheorem[style=enclosed]{defn}{Definition}
\mdtheorem[style=enclosed]{notn}{Notation}
\mdtheorem[style=enclosed]{obs}{Observation}
\mdtheorem[style=enclosed]{eg}{Example}
\mdtheorem[style=enclosed]{cor}{Corollary}
\mdtheorem[style=enclosed]{note}{Note}

% \let\thetheorem=\relax
% \let\thelemma=\relax
% \let\theclaim=\relax
% \let\theques=\relax
% \let\thedefn=\relax
% \let\thenotn=\relax
% \let\theobs=\relax
% \let\thecor=\relax
% \let\thenote=\relax

% \renewcommand\qedsymbol{$\blacksquare$}
\newcommand{\nl}{\vspace{0.2cm}\\}
\newcommand{\ol}{\overline}
\newcommand{\mc}{\mathcal}
\newcommand{\mi}{\mathit}
\newcommand{\mf}{\mathbf}
\newcommand{\mb}{\mathbb}
\newcommand{\R}{\mathbb{R}}
\newcommand{\OPT}{\mathbf{OPT}}
\newcommand{\ALG}{\mathbf{ALG}}
\renewcommand{\L}{\mc{L}}
\newcommand{\changesto}{\vdash}
\newcommand\Vtextvisiblespace[1][.3em]{%
    \mbox{\kern.06em\vrule height.3ex}%
    \vbox{\hrule width#1}%
    \hbox{\vrule height.3ex}
}
\newcommand{\blank}{{\Vtextvisiblespace[0.7em]}}
\newcommand{\leftend}{\triangleright}
\newcommand{\comp}{\overline}

\newcommand{\incfig}[1]{%
    \def\svgwidth{\columnwidth}
    \import{./figures/}{#1.pdf_tex}
}
\pdfsuppresswarningpagegroup=1

\title{\textbf{Approximation Algorithms Lecture 6}}
\date{}

%\section{Recap}
%
%\section{Definitions}
%
%\begin{defn}
%\end{defn}
%
%\section{Content}
%
%\begin{theorem}
%\end{theorem}
%\begin{proof}
%\end{proof}
%
%\begin{ques}
%\end{ques}
%
%\begin{eg}
%\end{eg}
%
%\begin{claim}
%\end{claim}

\begin{document}
\maketitle
\tableofcontents

\section{Recap}

Steiner tree problem

\section{Content}

\subsection{Travelling Salesman Problem (TSP)}

Variant 1: Given an undirected graph $G = (V, E)$ and a function $\ell : E \to \R^+$, find a cycle (need not be simple) in $G$ which visits all vertices and has the smallest length.\nl
Variant 2: Given a complete graph $G$ on $n$ vertices, and a function $\ell : E \to \R^+$, find a simple cycle in $G$ which visits all vertices and has the smallest length.\nl
Both variants are interreducible to one another.

\begin{claim}
    TSP is NP-hard.
\end{claim}

\begin{proof}

    For Variant 1:

    Claim: Hamiltonian Cycle $<_P$ TSP (i.e., a polynomial time algorithm for TSP gives a polynomial time algorithm for Hamiltonian cycle).\nl
    Recall that the HCP asks for a simple cycle which includes all vertices in a graph. Define the input to TSP as $(G, \ell)$ where $\ell(e) = 1$ if $e \in E$. If the answer to the TSP is $n$, then
    the graph has a Hamiltonian cycle. If more, then it doesn't.\nl

    For Variant 2:
    Replace the condition with $\ell(e) = 1$ if $e \in E$, $2$ otherwise.
\end{proof}

\begin{claim}
    It is NP-hard to approximate TSP to within any approximation factor $\alpha$.
\end{claim}

\begin{proof}
    We shall show that if there exists a polynomial time $\alpha$-approximation for TSP, then there is a polynomial time algorithm for HCP.\nl
    Call $\alpha$-TSP the algorithm which gives a solution within $\alpha$ times the optimum.\nl
    Let $G' = (V, V \times V)$, and $\ell(e) = 1$ if $e \in E$ and $\alpha n$ if $e \not\in E$. If $G$ is Hamiltonian, then an optimal solution to $(G', \ell)$ has value = $n$. So the solution
    returned by the $\alpha$-TSP has length $\le \alpha n$. So this solution can't have any edge that is not in $E$. Hence this solution is a Hamiltonian cycle in $G$ and has value $n$ (which is in
    fact the minimum possible).
    Hence, if the $\alpha$-TSP algorithm doesn't return $n$, then there is no solution.
    And if the $\alpha$-TSP algorithm returns $n$, we are clearly done (trivial bounding).
\end{proof}

\textbf{Gap Instance}: An instance where the optimum solution and the second-best solution have a large gap (say $\gamma$). Running an $\alpha$-competitive algorithm gives an answer within
$\alpha$ of the optimal answer, however if $\gamma > \alpha$, the algorithm is forced to output the best solution, which is why these instances can be important.\nl

Note that this reduction breaks down when we require that $\ell$ is a metric.

\subsection{Metric TSP}

The same problem as above, but $\ell$ is a metric.

\end{document}
