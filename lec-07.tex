\documentclass[a4paper]{article}
\usepackage[english]{babel}
\usepackage[a4paper,top=2cm,bottom=2cm,left=2cm,right=2cm,marginparwidth=1.75cm]{geometry}
\usepackage{amsmath}
\usepackage{amsfonts}
% \usepackage{amsthm}
\usepackage{amssymb}
\usepackage{graphicx}
\usepackage[colorinlistoftodos]{todonotes}
\usepackage[colorlinks=true, allcolors=blue]{hyperref}
\usepackage{import}
\usepackage{pdfpages}
\usepackage{transparent}
\usepackage{xcolor}
\usepackage{algorithmicx}
\usepackage{algpseudocode}

\usepackage{thmtools}
\usepackage{enumitem}
\usepackage[framemethod=TikZ]{mdframed}

\usepackage{xpatch}

\usepackage{boites}
\makeatletter
\xpatchcmd{\endmdframed}
{\aftergroup\endmdf@trivlist\color@endgroup}
{\endmdf@trivlist\color@endgroup\@doendpe}
{}{}
\makeatother

%\usepackage[poster]{tcolorbox}
%\allowdisplaybreaks
%\sloppy

\usepackage[many]{tcolorbox}

\xpatchcmd{\proof}{\itshape}{\bfseries\itshape}{}{}

% to set box separation
\setlength{\fboxsep}{0.8em}
\def\breakboxskip{7pt}
\def\breakboxparindent{0em}

\newenvironment{proof}{\begin{breakbox}\textit{Proof.}}{\hfill$\square$\end{breakbox}}
\newenvironment{ans}{\begin{breakbox}\textit{Answer.}}{\end{breakbox}}
\newenvironment{soln}{\begin{breakbox}\textit{Solution.}}{\end{breakbox}}

% \tcolorboxenvironment{proof}{
%     blanker,
%     before skip=\topsep,
%     after skip=\topsep,
%     borderline={0.4pt}{0.4pt}{black},
%     breakable,
%     left=12pt,
%     right=12pt,
%     top=12pt,
%     bottom=12pt,
% }
%
% \tcolorboxenvironment{ans}{
%     blanker,
%     before skip=\topsep,
%     after skip=\topsep,
%     borderline={0.4pt}{0.4pt}{black},
%     breakable,
%     left=12pt,
%     right=12pt,
% }

\mdfdefinestyle{enclosed}{
    linecolor=black
    ,backgroundcolor=none
    ,apptotikzsetting={\tikzset{mdfbackground/.append style={fill=gray!100,fill opacity=.3}}}
    ,frametitlefont=\sffamily\bfseries\color{black}
    ,splittopskip=.5cm
    ,frametitlebelowskip=.0cm
    ,topline=true
    ,bottomline=true
    ,rightline=true
    ,leftline=true
    ,leftmargin=0.01cm
    ,linewidth=0.02cm
    ,skipabove=0.01cm
    ,innerbottommargin=0.1cm
    ,skipbelow=0.1cm
}

\mdfsetup{%
    middlelinecolor=black,
    middlelinewidth=1pt,
roundcorner=4pt}

\setlength{\parindent}{0pt}

\mdtheorem[style=enclosed]{theorem}{Theorem}
\mdtheorem[style=enclosed]{lemma}{Lemma}[theorem]
\mdtheorem[style=enclosed]{claim}{Claim}[theorem]
\mdtheorem[style=enclosed]{ques}{Question}
\mdtheorem[style=enclosed]{defn}{Definition}
\mdtheorem[style=enclosed]{notn}{Notation}
\mdtheorem[style=enclosed]{obs}{Observation}
\mdtheorem[style=enclosed]{eg}{Example}
\mdtheorem[style=enclosed]{cor}{Corollary}
\mdtheorem[style=enclosed]{note}{Note}

% \let\thetheorem=\relax
% \let\thelemma=\relax
% \let\theclaim=\relax
% \let\theques=\relax
% \let\thedefn=\relax
% \let\thenotn=\relax
% \let\theobs=\relax
% \let\thecor=\relax
% \let\thenote=\relax

% \renewcommand\qedsymbol{$\blacksquare$}
\newcommand{\nl}{\vspace{0.2cm}\\}
\newcommand{\ol}{\overline}
\newcommand{\mc}{\mathcal}
\newcommand{\mi}{\mathit}
\newcommand{\mf}{\mathbf}
\newcommand{\mb}{\mathbb}
\newcommand{\R}{\mathbb{R}}
\newcommand{\OPT}{\mathbf{OPT}}
\newcommand{\ALG}{\mathbf{ALG}}
\renewcommand{\L}{\mc{L}}
\newcommand{\changesto}{\vdash}
\newcommand\Vtextvisiblespace[1][.3em]{%
    \mbox{\kern.06em\vrule height.3ex}%
    \vbox{\hrule width#1}%
    \hbox{\vrule height.3ex}
}
\newcommand{\blank}{{\Vtextvisiblespace[0.7em]}}
\newcommand{\leftend}{\triangleright}
\newcommand{\comp}{\overline}

\newcommand{\incfig}[1]{%
    \def\svgwidth{\columnwidth}
    \import{./figures/}{#1.pdf_tex}
}
\pdfsuppresswarningpagegroup=1

\title{\textbf{Approximation Algorithms Lecture 7}}
\date{}

%\section{Recap}
%
%\section{Definitions}
%
%\begin{defn}
%\end{defn}
%
%\section{Content}
%
%\begin{theorem}
%\end{theorem}
%\begin{proof}
%\end{proof}
%
%\begin{ques}
%\end{ques}
%
%\begin{eg}
%\end{eg}
%
%\begin{claim}
%\end{claim}

\begin{document}
\maketitle
\tableofcontents

\section{Recap}

Introduction to TSP.

\section{Content}

\subsection{Travelling Salesman Problem (TSP)}

Variant 1: Given an undirected graph $G = (V, E)$ and a function $\ell : E \to \R^+$, find a cycle (need not be simple) in $G$ which visits all vertices and has the smallest length.\nl
Variant 2: Given a complete graph $G$ on $n$ vertices, and a function $\ell : E \to \R^+$, find a simple cycle in $G$ which visits all vertices and has the smallest length.\nl
Both variants are interreducible to one another.\nl
\begin{claim}
    It is NP-hard to approximate TSP to within any approximation factor $\alpha$.
\end{claim}

\subsection{Metric TSP}

The same problem as above, but $\ell$ is a metric. Note that the proof in the previous lecture breaks down, so it might be the case that we can get an approximation (remember that it needs to run in
polynomial time), which is why we impose the restriction that $\ell$ is a metric, which is also the case in most practical situations.\nl
Id est, the following are true:
\begin{enumerate}
    \item Triangle inequality: $\ell(u, v) \le \ell(u, w) + \ell(w, v) \forall u, v, w \in V$
    \item Symmetric: $\ell(u, v) = \ell(v, u)$ $\forall u, v \in V$.
    \item $\ell(u, u) = 0$ $\forall u, v \in V$.
\end{enumerate}

This problem is also called the metric TSP.\nl

\subsection{2-approximation for metric TSP}

We have a 2-approximation for the metric TSP as follows:

\begin{enumerate}
    \item Find a minimum spanning tree in $G$.
    \item Consider an Euler tour of this spanning tree (skipping vertices which are duplicated).
\end{enumerate}

\begin{proof}
    Total length of the tour is $\le 2 \cdot \text{length of the MST}$. Now we note that the length of the MST is a lower bound on the optimum, since removing an edge of the optimum tour gives
    us a spanning tree, whose length is at least the length of the MST. So we have length of the MST $\le \OPT$, which in turn implies that $\ALG \le 2 \cdot \OPT$, as needed. Note that we can
    remove the max edge from $\OPT$ which gives us a $2 \cdot \left(1 - \frac{1}{n}\right)$ approximation.
\end{proof}

\subsection{Christofides algorithm}
We also have a $\frac{3}{2}$-approximation for the metric TSP as follows.\nl
So what we did was that we took 2 copies of each edge of the MST to create an Eulerian subgraph, and short-circuited the Eulerian subgraph to obtain a TSP-tour. (A graph is Eulerian if there is a
cycle which visits every edge exactly once, which is equivalent to saying that the degree of every vertex is even and the graph is connected).\nl
We convert the MST into an Eulerian subgraph by picking a min-length perfect matching on odd degree vertices $T$ in the MST. Such a perfect matching exists since it's a complete graph and the number of odd degree
vertices is even.\nl
Now we shall do an Eulerian tour on this subgraph. The length of this tour is at most the total length of edges in the Eulerian subgraph which is the length of the MST plus the length of the
minimum perfect matching on $T$.\nl
Now consider the optimal tour. Consider two matchings (greedily choose alternate pairs of odd-degree vertices from the MST in the cycle). The total lengths of these two matchings is at most $\OPT$ due to
the triangle inequality, so the shorter of these two perfect matchings is at most $\OPT/2$. (How exactly to apply the triangle inequality).\nl
Note that the addition of simple cycle is what makes the non-metric TSP hard.

\end{document}
