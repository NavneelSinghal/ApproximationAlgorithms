\documentclass[a4paper]{article}
\usepackage[english]{babel}
\usepackage[a4paper,top=2cm,bottom=2cm,left=2cm,right=2cm,marginparwidth=1.75cm]{geometry}
\usepackage{amsmath}
\usepackage{amsfonts}
% \usepackage{amsthm}
\usepackage{amssymb}
\usepackage{graphicx}
\usepackage[colorinlistoftodos]{todonotes}
\usepackage[colorlinks=true, allcolors=blue]{hyperref}
\usepackage{import}
\usepackage{pdfpages}
\usepackage{transparent}
\usepackage{xcolor}
\usepackage{algorithmicx}
\usepackage{algpseudocode}

\usepackage{thmtools}
\usepackage{enumitem}
\usepackage[framemethod=TikZ]{mdframed}

\usepackage{xpatch}

\usepackage{boites}
\makeatletter
\xpatchcmd{\endmdframed}
{\aftergroup\endmdf@trivlist\color@endgroup}
{\endmdf@trivlist\color@endgroup\@doendpe}
{}{}
\makeatother

%\usepackage[poster]{tcolorbox}
%\allowdisplaybreaks
%\sloppy

\usepackage[many]{tcolorbox}

\xpatchcmd{\proof}{\itshape}{\bfseries\itshape}{}{}

% to set box separation
\setlength{\fboxsep}{0.8em}
\def\breakboxskip{7pt}
\def\breakboxparindent{0em}

\newenvironment{proof}{\begin{breakbox}\textit{Proof.}}{\hfill$\square$\end{breakbox}}
\newenvironment{ans}{\begin{breakbox}\textit{Answer.}}{\end{breakbox}}
\newenvironment{soln}{\begin{breakbox}\textit{Solution.}}{\end{breakbox}}

% \tcolorboxenvironment{proof}{
%     blanker,
%     before skip=\topsep,
%     after skip=\topsep,
%     borderline={0.4pt}{0.4pt}{black},
%     breakable,
%     left=12pt,
%     right=12pt,
%     top=12pt,
%     bottom=12pt,
% }
%
% \tcolorboxenvironment{ans}{
%     blanker,
%     before skip=\topsep,
%     after skip=\topsep,
%     borderline={0.4pt}{0.4pt}{black},
%     breakable,
%     left=12pt,
%     right=12pt,
% }

\mdfdefinestyle{enclosed}{
    linecolor=black
    ,backgroundcolor=none
    ,apptotikzsetting={\tikzset{mdfbackground/.append style={fill=gray!100,fill opacity=.3}}}
    ,frametitlefont=\sffamily\bfseries\color{black}
    ,splittopskip=.5cm
    ,frametitlebelowskip=.0cm
    ,topline=true
    ,bottomline=true
    ,rightline=true
    ,leftline=true
    ,leftmargin=0.01cm
    ,linewidth=0.02cm
    ,skipabove=0.01cm
    ,innerbottommargin=0.1cm
    ,skipbelow=0.1cm
}

\mdfsetup{%
    middlelinecolor=black,
    middlelinewidth=1pt,
roundcorner=4pt}

\setlength{\parindent}{0pt}

\mdtheorem[style=enclosed]{theorem}{Theorem}
\mdtheorem[style=enclosed]{lemma}{Lemma}[theorem]
\mdtheorem[style=enclosed]{claim}{Claim}[theorem]
\mdtheorem[style=enclosed]{ques}{Question}
\mdtheorem[style=enclosed]{defn}{Definition}
\mdtheorem[style=enclosed]{notn}{Notation}
\mdtheorem[style=enclosed]{obs}{Observation}
\mdtheorem[style=enclosed]{eg}{Example}
\mdtheorem[style=enclosed]{cor}{Corollary}
\mdtheorem[style=enclosed]{note}{Note}

% \let\thetheorem=\relax
% \let\thelemma=\relax
% \let\theclaim=\relax
% \let\theques=\relax
% \let\thedefn=\relax
% \let\thenotn=\relax
% \let\theobs=\relax
% \let\thecor=\relax
% \let\thenote=\relax

% \renewcommand\qedsymbol{$\blacksquare$}
\newcommand{\nl}{\vspace{0.2cm}\\}
\newcommand{\ol}{\overline}
\newcommand{\eps}{\varepsilon}
\newcommand{\mc}{\mathcal}
\newcommand{\mi}{\mathit}
\newcommand{\mf}{\mathbf}
\newcommand{\mb}{\mathbb}
\newcommand{\R}{\mathbb{R}}
\newcommand{\Z}{\mathbb{Z}}
\newcommand{\OPT}{\mathbf{OPT}}
\newcommand{\ALG}{\mathbf{ALG}}
\renewcommand{\L}{\mc{L}}
\newcommand{\changesto}{\vdash}
\newcommand\Vtextvisiblespace[1][.3em]{%
    \mbox{\kern.06em\vrule height.3ex}%
    \vbox{\hrule width#1}%
    \hbox{\vrule height.3ex}
}
\newcommand{\blank}{{\Vtextvisiblespace[0.7em]}}
\newcommand{\leftend}{\triangleright}
\newcommand{\comp}{\overline}

\newcommand{\incfig}[1]{%
    \def\svgwidth{\columnwidth}
    \import{./figures/}{#1.pdf_tex}
}
\pdfsuppresswarningpagegroup=1

\title{\textbf{Approximation Algorithms Lecture 13}}
\date{}

%\section{Recap}
%
%\section{Definitions}
%
%\begin{defn}
%\end{defn}
%
%\section{Content}
%
%\begin{theorem}
%\end{theorem}
%\begin{proof}
%\end{proof}
%
%\begin{ques}
%\end{ques}
%
%\begin{eg}
%\end{eg}
%
%\begin{claim}
%\end{claim}

\begin{document}
\maketitle
\tableofcontents

\section{Recap}

Fully polynomial time approximation scheme (FPTAS) for knapsack problem

\section{Content}

\subsection{APTAS for bin packing}

We will exhibit an APTAS for bin packing with the solution $\le (1 + \eps) \OPT + 1/\eps^2 + 2$.\nl
We will first write an integer program for bin packing.\nl
Given $n_i$ objects of size $s_i$, $1 \le i \le k$, and $\sum_i n_i = m$.\nl
A configuration $c$ is a $k$-tuple $(o_1^c, o_2^c, \ldots, o_k^c)$ such that $\sum_i o_i s_i \le 1$, and $o_i \in \Z_{\ge 0}$.\nl
Let $x_c$ be the number of bins which have configuration $c$. Then $x_c \in \Z_{\ge 0}$.\nl
Then we need to minimize $\sum_{c} x_c$.\nl
There is another constraint: $\sum_{c} x_c o_i^c \ge n_i$ for all $1 \le i \le k$, which is the integer program.\nl
Relax the integrality constraint on $x_c$ to $x_c \ge 0$, and this will give us a linear program.\nl
Note that in a linear program, the optimal answer is at a vertex.
Let the number of different configurations be $\alpha$ (this is also the "dimension" of the LP). Then the optimal solution lies at the intersection of $\alpha$ hyperplanes (inequalities).\nl
We have $\alpha + k$ inequalities in all and at a vertex, we set $\alpha$ inequalities as equalities. At least $\alpha - k$ configurations are set to $0$, and so at most $k$ configurations have
non-zero $x_c$.\nl
Our algorithm then becomes:
\begin{enumerate}
    \item Solve the LP and round up all $x_c$ variables.
\end{enumerate}
Note that solution is $\le$ LP value + $k$ (since rounding will increase by at most $k$), which is at most $\OPT + k$.\nl
Note that $k$ is large, and we want to reduce it.\nl
We will group objects to reduce number of distinct sizes as follows: remove all objects of size $\le \eps$, order objects by decreasing size and form groups of $g$ objects.\nl
Redefine $k$ as the number of distinct sizes now. (Verify this later on).
Round up each object in a group to the size of the largest object in that group.\nl
Let $I'$ be this new instance.\nl
If we run the earlier algorithm on $I'$ then we get a solution with number of bins $\le \OPT' + \lceil\frac{k}{g}\rceil$.\nl
In going from $I$ to $I'$ we only increased the object sizes, hence the solution for $I'$ can also pack the objects in $I$.\nl
How does $\OPT'$ relate to $\OPT$?\nl
\begin{claim}
    $\OPT' \le \OPT + g$
\end{claim}
\begin{proof}
    Consider a solution which requires $\OPT$ bins to pack objects of $I$.\nl
    The objects of $I'$, except those of $g_1$ can be packed in the spaces occupied by the objects of $I$. To see this, we can just construct a mapping from $g_i$ to $g_{i-1}$, and put object $f(o)$
    where $o$ is put in $I$.\nl
    Now each group 1 object can be packed in a separate bin, so we are done.
\end{proof}
So all objects can be packed in $\OPT' + g + \lceil \frac{k}{g} \rceil$ bins.\nl
Here $k$ is the number of distinct sizes obtained after throwing away all objects of size $\le \eps$.
We will choose $g = \lceil \eps^2 k \rceil$, i.e., $g \le \eps^2 k + 1$.\nl
So the solution we get is $\le \OPT + \eps^2 k + 1 + \lceil\frac{1}{\eps^2}\rceil \le \OPT + \eps^2 k + \frac{1}{\eps^2} + 2$.\nl
We have $\eps \le \frac{\OPT}{k}$ since we have at least $k$ objects of size $\eps$, so the total volume is $\ge \eps k$, whence the total number of bins is at least $\lceil \eps k \rceil$. Then this gets us an upper bound of
$\left(1 + \frac{1}{\eps}\right)\OPT + \frac{1}{\eps^2} + 2$.\nl
How to handle small objects of size $|le \eps$?\nl
\begin{enumerate}
    \item Consider small objects in any order.
    \item For each object, check if it can fit into one of the bins. If not, open a new bin.
\end{enumerate}
Note that if I open a new bin, then each bin has objects of size at least $1 - \eps$. Suppose we had $r$ bins opened earlier, then the total size of objects is at least $(1 - \eps)r$, which
means that $\OPT \ge (1 - \eps) r$.\nl
This means that $r \le \frac{\OPT}{1 - \eps}$.\nl
So number of bins in my solution is $r + 1 \le \frac{\OPT}{1 - \eps} + 1 \le \OPT(1 + \eps) + 1$. (fix this later on using $\OPT'$).\nl

\end{document}
