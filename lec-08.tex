\documentclass[a4paper]{article}
\usepackage[english]{babel}
\usepackage[a4paper,top=2cm,bottom=2cm,left=2cm,right=2cm,marginparwidth=1.75cm]{geometry}
\usepackage{amsmath}
\usepackage{amsfonts}
% \usepackage{amsthm}
\usepackage{amssymb}
\usepackage{graphicx}
\usepackage[colorinlistoftodos]{todonotes}
\usepackage[colorlinks=true, allcolors=blue]{hyperref}
\usepackage{import}
\usepackage{pdfpages}
\usepackage{transparent}
\usepackage{xcolor}
\usepackage{algorithmicx}
\usepackage{algpseudocode}

\usepackage{thmtools}
\usepackage{enumitem}
\usepackage[framemethod=TikZ]{mdframed}

\usepackage{xpatch}

\usepackage{boites}
\makeatletter
\xpatchcmd{\endmdframed}
{\aftergroup\endmdf@trivlist\color@endgroup}
{\endmdf@trivlist\color@endgroup\@doendpe}
{}{}
\makeatother

%\usepackage[poster]{tcolorbox}
%\allowdisplaybreaks
%\sloppy

\usepackage[many]{tcolorbox}

\xpatchcmd{\proof}{\itshape}{\bfseries\itshape}{}{}

% to set box separation
\setlength{\fboxsep}{0.8em}
\def\breakboxskip{7pt}
\def\breakboxparindent{0em}

\newenvironment{proof}{\begin{breakbox}\textit{Proof.}}{\hfill$\square$\end{breakbox}}
\newenvironment{ans}{\begin{breakbox}\textit{Answer.}}{\end{breakbox}}
\newenvironment{soln}{\begin{breakbox}\textit{Solution.}}{\end{breakbox}}

% \tcolorboxenvironment{proof}{
%     blanker,
%     before skip=\topsep,
%     after skip=\topsep,
%     borderline={0.4pt}{0.4pt}{black},
%     breakable,
%     left=12pt,
%     right=12pt,
%     top=12pt,
%     bottom=12pt,
% }
%
% \tcolorboxenvironment{ans}{
%     blanker,
%     before skip=\topsep,
%     after skip=\topsep,
%     borderline={0.4pt}{0.4pt}{black},
%     breakable,
%     left=12pt,
%     right=12pt,
% }

\mdfdefinestyle{enclosed}{
    linecolor=black
    ,backgroundcolor=none
    ,apptotikzsetting={\tikzset{mdfbackground/.append style={fill=gray!100,fill opacity=.3}}}
    ,frametitlefont=\sffamily\bfseries\color{black}
    ,splittopskip=.5cm
    ,frametitlebelowskip=.0cm
    ,topline=true
    ,bottomline=true
    ,rightline=true
    ,leftline=true
    ,leftmargin=0.01cm
    ,linewidth=0.02cm
    ,skipabove=0.01cm
    ,innerbottommargin=0.1cm
    ,skipbelow=0.1cm
}

\mdfsetup{%
    middlelinecolor=black,
    middlelinewidth=1pt,
roundcorner=4pt}

\setlength{\parindent}{0pt}

\mdtheorem[style=enclosed]{theorem}{Theorem}
\mdtheorem[style=enclosed]{lemma}{Lemma}[theorem]
\mdtheorem[style=enclosed]{claim}{Claim}[theorem]
\mdtheorem[style=enclosed]{ques}{Question}
\mdtheorem[style=enclosed]{defn}{Definition}
\mdtheorem[style=enclosed]{notn}{Notation}
\mdtheorem[style=enclosed]{obs}{Observation}
\mdtheorem[style=enclosed]{eg}{Example}
\mdtheorem[style=enclosed]{cor}{Corollary}
\mdtheorem[style=enclosed]{note}{Note}

% \let\thetheorem=\relax
% \let\thelemma=\relax
% \let\theclaim=\relax
% \let\theques=\relax
% \let\thedefn=\relax
% \let\thenotn=\relax
% \let\theobs=\relax
% \let\thecor=\relax
% \let\thenote=\relax

% \renewcommand\qedsymbol{$\blacksquare$}
\newcommand{\nl}{\vspace{0.2cm}\\}
\newcommand{\ol}{\overline}
\newcommand{\mc}{\mathcal}
\newcommand{\mi}{\mathit}
\newcommand{\mf}{\mathbf}
\newcommand{\mb}{\mathbb}
\newcommand{\R}{\mathbb{R}}
\newcommand{\OPT}{\mathbf{OPT}}
\newcommand{\ALG}{\mathbf{ALG}}
\renewcommand{\L}{\mc{L}}
\newcommand{\changesto}{\vdash}
\newcommand\Vtextvisiblespace[1][.3em]{%
    \mbox{\kern.06em\vrule height.3ex}%
    \vbox{\hrule width#1}%
    \hbox{\vrule height.3ex}
}
\newcommand{\blank}{{\Vtextvisiblespace[0.7em]}}
\newcommand{\leftend}{\triangleright}
\newcommand{\comp}{\overline}

\newcommand{\incfig}[1]{%
    \def\svgwidth{\columnwidth}
    \import{./figures/}{#1.pdf_tex}
}
\pdfsuppresswarningpagegroup=1

\title{\textbf{Approximation Algorithms Lecture 8}}
\date{}

%\section{Recap}
%
%\section{Definitions}
%
%\begin{defn}
%\end{defn}
%
%\section{Content}
%
%\begin{theorem}
%\end{theorem}
%\begin{proof}
%\end{proof}
%
%\begin{ques}
%\end{ques}
%
%\begin{eg}
%\end{eg}
%
%\begin{claim}
%\end{claim}

\begin{document}
\maketitle
\tableofcontents

\section{Recap}

2 and 3/2 approximations to Metric TSP.

\section{Content}

\subsection{Facility location problems - $k$-center problem}

Given $n$ points in a plane, locate $k$ centers at these points so that the maximum (Euclidean) distance of any point from its closest center is minimized.\nl
Given a location of the facilities, the cost of this solution can be easily computed.\nl
A solution is specified by the location of the facilities.\nl

A similar problem is the $k$-median problem: open $k$ facilities so that the sum of distances of clients to closest facility is minimized.\nl

Note that the distances are Euclidean, so the triangle inequality is satisfied. This problem can also be given to us on a graph, (unweighted satisfies triangle inequality as well).\nl
We show a 2-approximation for the $k$-center problem when distances form a metric.
\begin{enumerate}
    \item Make a guess on the value (max distance a client has to travel) of the optimal solution, let it be $r^*$.
    \item $C$ is the initial set of vertices. $S$ is an initially empty set which will hold the solution in the end.
    \item Pick a vertex $v \in C$ and add it to $S$.
    \item Remove all vertices of $C$ which are within distance $2 r^*$ from $v$.
    \item Repeat the last two steps till $C$ is empty.
\end{enumerate}

\begin{proof}
    Firstly note that if $r^*$ is a correct guess (i.e., it is an upper bound on the answer), then we have $|S| \le k$. To prove this, consider the following argument. At each point, the circle
    of radius $r^*$ centered around the nearest center is going to get removed, so the next point taken will have closest center not equal to this vertex.\nl
    For another proof, draw a ball of radius
    $r^*$ around each of these picked vertices. These balls are all disjoint, since otherwise the one center belongs to a disc of radius $2r^*$ around another center (and vice-versa), so while
    processing the point picked before, we would have removed the point taken later on. So any solution with value $r^*$ has to pick a vertex in each ball. So the number of centers is at least $|S|
    > k$ which is a contradiction (since suppose there is no center in a ball, then since optimal answer is $\le r^*$, then the center of this ball is outside the actual ball of radius $r$ (not
    $r^*$) around the optimal center inside any other ball).
\end{proof}

Now note that if our choice of $r^*$ is very small, then we will pick more than $k$ centers. So the right choice of $r^*$ is the smallest value for which we pick $k$ or less centers.\nl
Note that the set of possible (optimal) $r^*$ is equal to the distance between two points (the min possible radius for a given $|S|$). So we can iterate over all possible distances, and check for $r^*$ at each iteration. Suppose $r$
is the smallest value at which our algorithm picked $\le k$ centers. Then $\OPT \ge r$. Value of our solution is $\le 2r$, so our solution is $\le 2 \cdot \OPT$.\nl
Note that this proof works for a metric as well.\nl
\begin{claim}
    It is NP-hard to approximate $k$-center to a factor less than $2$.
\end{claim}
\begin{proof}
    Suppose there is a $2 - \epsilon$ approximation for $k$-center problem. We claim that we can solve vertex cover using this problem. Given an instance $(V, E), k$ of vertex cover, convert it
    to $(V, V \times V, d : V \times V \to \R), k$ where $d(u, v) = 1$ if $(u, v) \in E$ and $2$ otherwise.\nl
    Note that the optimal solution of this formed instance has value 1 or value 2. The answer is 1 iff $G$ has a vertex cover of size $k$ - this is a wrong claim. Instead of vertex cover, use
    dominating set.
\end{proof}

\end{document}
